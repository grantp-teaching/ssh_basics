\documentclass{pgnotes}

\title{SSH lab}

\begin{document}

\maketitle

\section{Basic connection}

You will be given details of server(s) by the lecturer, to include name/IP and username.
Make an SSH connection to each server using the \texttt{ssh} command.
Note down what operating system each server appears to be running. 

\subsection{Basic connection using PuTTY}

Use the PuTTY application to connect to the server instead of the SSH command.
Take time to modify the fonts/colours to make the text more readable.
Save the configuration for next time.

Re-connect using your saved configuration.

\section{Key setup}

Following the notes, use \texttt{ssh-keygen} to set up a public/private key pair.

Use \texttt{check\_key\_exists.ps1} to confirm that your keys are in the correct place.

\subsection{Submit key}

Submit your PUBLIC key as shown by the lecturer.

\section{Key-based connection}

Use the key to login as before.
Note the lack of a password prompt.


\end{document}

